Wearable robotic devices are intended as a means to augment human economy, strength, and endurance. Over the past decade, a number of devices have been developed for reducing the metabolic cost of walking for able-bodied individuals \citep{Malcolm2013, Mooney2016, Caputo2014,  Panizzolo2016}. With tethered and portable hardware platforms having advanced considerably \citep{Malcolm2013, Lee2018, Panizzolo2015}, it is now possible to accurately control settings such as the timing and magnitude of the delivered torque \citep{Malcolm2013, Collins2015, Ding2017, Kim2015, Kim2017inv, Lee2016}. Increased attention has been directed towards how control strategy and control parameters influence overall system performance \citep{Caputo2014, Kim2015, Kim2017inv, Quinlivan2017}. Traditionally, control parameters have been tuned manually by researchers with expert knowledge on both the device and biomechanics of human walking or through exploring the parameter space in a systematic method sweep \citep{Caputo2015, Quinlivan2017}. However, despite these efforts, several studies have shown significant inter-subject variability in the observed metabolic benefit for a fixed parameter settings~\citep{Quesada2016}. Being able to automatically recover the optimal parameter setting on an individual basis is therefore a fundamental component in designing effective wearable robotic devices.

Recently, several groups have explored different formulations of \emph{human-in-the-loop} (HIL) optimization that offer the possibility to avoid conventional exhaustive search~\citep{Koller2016, Zhang2017, Ding2018, Kim2017}. It uses instantaneous energetic cost~\citep{Selinger2014}, an estimation of steady-state metabolic cost, as an objective measurement for automatically optimizing parameter settings~\citep{Koller2016, Felt2015, Zhang2017, Ding2018}. Previous research has applied this HIL optimization approach to find optimal onset actuation timing of a bilateral pneumatic ankle exoskeleton by automatically adjusting a single control parameter~\citep{Koller2016}. A recent study with an ankle exoskeleton~\citep{Zhang2017} and hip exosuit~\citep{Ding2018} showed the potential to achieve larger metabolic benefits by simultaneously optimizing multiple parameters using Bayesian optimization.

Bayesian optimization is a sequential optimization strategy that has been used in a variety of applications from hyperparameter tuning for machine learning algorithms~\citep{NIPS2012_4522,pmlr-v22-mahendran12} to portfolio allocation \citep{1009.5419} and experimental design \citep{Brochu2009}. As it is a general framework to optimize noisy black box functions, many variations have been introduced to account for different use cases. In particular, an area of interest has been in tackling the noisiness of data samples \citep{rue2009,1603.02038,1410.7172} and the myopia of the algorithm \citep{NIPS2016_6188,1510.06299}. Lookahead approaches are typically framed in the context of an iteration budget, decoupled from the actual data sampling process. However, in the HIL problem there is a limited time constraint rather than an iteration constraint, and data acquisition exhibits a direct tradeoff between noisiness and elapsed time. We frame this tradeoff as a stopping problem before informing the Bayesian optimization algorithm of a new training sample.