\subsection*{Summary}
This thesis aims to introduce a stopping criteria for measurement time within HIL optimization. A metabolic estimator based off the Unscented Kalman Filter algorithm was designed and has effectively shown the ability to track the instantaneous energetic cost online with varying amounts of data. Two methods were developed for early stopping: a simple $\sigma$-offset threshold strategy and a bandit process strategy based off the Gittins Index. Depending on the noisiness of the estimator, the Gittins approach can provide a higher fault tolerance for noisy observations. Bayesian optimization with a Gaussian process prior was used as an overall design framework for selecting parameter settings to sample from. Two trials were then conducted using the Gittins approach to optimize over six hip parameters and similar metabolic reduction was achieved compared to a previous study with the same subjects. While the previous study had a shorter duration time, it was only a two-parameter study and ample time was spent prior to the subject tests isolating the parameters individually to find the two with highest variability (peak and offset timing). In our case, we were able to find similar results without the need for an extensive pre-trial exploration phase. Interestingly we also found a lower peak force to provide similar metabolic reduction in each of the two subjects. Having a lower peak force provides two benefits: lower battery requirements on the suit and reduced strain on the harness and subsequent discomfort to the user.

\subsection*{Challenges and Future Work}
There were a number of implementation choices that can be improved upon in future work. While collecting ample metabolic data is a time-constrained challenge, in the future a prior metabolic landscape could potentially be developed as a mean function for the Gaussian process. With respect to the Bayesian optimization, accounting for the early stopping of parameter settings led to fixed noise hyperparameters based off the metabolic estimator. Alternatively, a fully heteroscedastic approach could be experimented with that involves fitting output noise to a secondary Gaussian process. For the metabolic estimator, the high subject-to-subject variability in measurements requires some optimization for the covariance matrices. Given Bayesian optimization has an exploration phase, tuning the estimator based on this data should improve reliability. Similarly, different methods for tuning the risk tolerances of the stopping algorithms could be analyzed through additional subject tests. Two subject trials is certainly not enough, and many more subject tests should be carried out for a better analysis of the stopping algorithms in practice. Finally, an area of current interest is multi-joint optimization using both ankle and hip actuation \citep{1298569}. This would be a natural use case for the stopping algorithm as the number of optimization parameters could potentially double. Current pilot tests are being run optimizing four parameters: peak timing in both the hip and ankle, onset timing in the ankle, and offset timing in the hip.